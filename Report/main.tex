\documentclass[letterpaper, 11pt]{article}
\usepackage[utf8]{inputenc}
\usepackage{mathptmx}
\usepackage{natbib}
\usepackage[top=1in, bottom=1.25in, left=1.25in, right=1.25in]{geometry}
\usepackage{enumitem}
\usepackage[usenames,dvipsnames]{xcolor}
\usepackage{hyperref}
\hypersetup{colorlinks=true,allcolors=MidnightBlue,pdfauthor={surajss2}}
\usepackage{titlesec}
\usepackage{graphicx}
\usepackage{wrapfig}
\usepackage{enumitem}
\usepackage{array}
\usepackage{booktabs}
\usepackage{comment}
\usepackage{csquotes}
\usepackage{microtype}
\usepackage[ragged]{footmisc}

\title{CS 421: Final Report}
\author{Suraj Singh \\ NetId: surajss2 }

\begin{document}

\maketitle

\section{Overview}\label{sec:overview}
%%% Describe the motivation, goals, and broad accomplishments of your project in general terms.
\section{Implementation}\label{sec:impl}
%%%A brief description of the important aspects of your implementation, in  terms of (a) the major tasks or capabilities of your code; (b)  components of the code; (d) status of the project – what works well,  what works partially, and and what is not implemented at all. You MUST  compare these with your original proposed goals in the project proposal.
\section{Tests}\label{sec:tests}
%%% Coming up with appropriate tests is one of the most important part of  good software development. Your tests should include unit tests, feature  tests, and larger test codes. Give a short description of the tests  used, performance results if appropriate (e.g., memory consumption for  garbage collection) etc. Be sure to explain how these tests exercise the  concept(s) you've implemented.


\section{Listing}\label{sec:code}
%%% Coming up with appropriate tests is one of the most important part of  good software development. Your tests should include unit tests, feature  tests, and larger test codes. Give a short description of the tests  used, performance results if appropriate (e.g., memory consumption for  garbage collection) etc. Be sure to explain how these tests exercise the  concept(s) you've implemented.

\end{document}
